% Time-stamp: <Thu 2022-02-03 14:58 svarrette>
% ==============================================================================
% cv-varrette-en.tex -- main LaTeX file of my personal CV
%
% Copyright (c) 2009-2011 Sebastien Varrette <Sebastien.Varrette@uni.lu>
% .             http://varrette.gforge.uni.lu
%
% .     ______     __          ____    __     __                  _   _
% .    / ___\ \   / /         / ___|   \ \   / /_ _ _ __ _ __ ___| |_| |_ ___
% .   | |    \ \ / /   _____  \___ \    \ \ / / _` | '__| '__/ _ \ __| __/ _ \
% .   | |___  \ V /   |_____|  ___) |    \ V / (_| | |  | | |  __/ |_| ||  __/
% .    \____|  \_/            |____(_)    \_/ \__,_|_|  |_|  \___|\__|\__\___|
%
% ==============================================================================
% This work is licensed under the CC-by-nc-sa 3.0 licence (see LICENCE file).
% For more details, visit:  http://creativecommons.org/licenses/by-nc-sa/3.0/
%
% For more info about this work, see the README file
% ==============================================================================
\documentclass{cv}

\usepackage{_style}


\begin{document}
% ---------- Header -----------
\begin{chapeau}
  \begin{adresse}
    {\Large\textbf{Sebastien VARRETTE, PhD}}\\
    \ligne\\
    \textbf{Research Scientist, Head Research Computing and HPC operations}\\
    \textbf{Management, Security and Performance of HPC systems}\\
    %\textbf{15 years of experience in Linux HPC systems}\\
    \ligne\\
    Phone: +33(0)6~74~57~90~05\\
    E-mail:    \url{Sebastien.Varrette@uni.lu}\\
    Home page: \url{http://varrette.gforge.uni.lu}\\
    GPG Key ID: \href{https://pgp.mit.edu/pks/lookup?op=vindex&search=0x5D08BCDD4F156AD7}{5D08BCDD4F156AD7}
  \end{adresse}
  \begin{etatcivil}
    %\includegraphics[width=0.2\textwidth]{Images/cv.jpg}
    \photo{Images/sv.jpg} %MyPhoto}

    Born on November 27th, 1979 (in France)\\
    Married (2004), two children (2007,2010)\\
  \end{etatcivil}
\end{chapeau}
% -----------------------------
\vspace*{0.5em}
\noindent\rule{\textwidth}{0.4pt}

\emph{Short bio:}
With more than 14 years of postdoctoral and team management experience, Dr. Varrette is Research Scientist within the University of Luxembourg.
Expert in the deployment and management of High Performance Computing (HPC) systems, he is leading the University’s HPC and Big Data platform, and the associated expert team managing and supporting this facility.\\
In parallel, he is pursuing his research in the domains of the security and performance of parallel and distributed computing platforms, such as HPC, Cloud Computing or Data Analytics infrastructures.
His research contributions led to more than 80 publications in high-level scientific journals, or international conference proceedings while co-authoring 4 books. He has a strong involvement in the community with reviewing roles in impact journals, conferences organization (e.g., IEEE CloudCom) and the participation to more than 50 conference program committee.\\
Finally, he takes part for the management committee and represents Luxembourg within multiple EU HPC projects, such as \href{http://www.prace-ri.eu/}{PRACE} (acting Advisor), \href{https://www.castiel-project.eu/}{CASTIEL}, \href{http://www.etp4hpc.eu/}{ETP4HPC} or several COST actions. He has also concrete contributions in several strategic projects linked to HPC developed with multiple key decision makers in the Luxembourg context, either from the private sector or at the governmental level (Ex: EuroHPC MeluXina Supercomputer). He's also acting as HPC expert for the European Commission in the Enhanced Regional EU-ASEAN Dialogue Instrument (E-READI) program.


%\iffullcv{
\noindent\rule{\textwidth}{0.4pt}

\vspace{1em}
\begin{tabular}{ll}
  \textsc{Technical / Management} & \acf{HPC}, \acf{BD} \& Cloud technologies.
  \\
  \textsc{Expertise:}& Managing large-scale \ac{HPC} and Data analytics systems since 2003
                       (Linux environment)
  \\
                                  & \textbf{HPC Team Leader since 2007} (Head, Uni.lu HPC facility)
                                  % \offset \offset \underline{\textit{Keywords:}} HPC, Debian, Xen, Puppet,
                                  % OpenLDAP, git\\
  \\
  \textsc{Main research domains}: & Security and Performance Evaluation of Distributed
                                    Computing Platforms \\ % (clusters, grids or clouds).\\
  \iffullcv{
                                  & \emph{Relevant contributions per domains}: \\
                                  % & Main achievements:\\
                                  & \offset HPC, Perf. \& Energy efficiency: \cvcite{DVB_SPECTS08,JVOB_EELSDS13,VGPBP_SBACPAD13,VPGBB_ICPP14,EVB_CLOUD16,OLADMFGGLLRMOPPPSSV_NESUS_COSTBook_Chap5,OV_NESUS_COSTBook_SubChap5,EVPB_TCC_19,VPKDB_PPAM19,MVPPB_CCGRID20,PVB_APF21,VKPKPCB_PEARC21} \\
                                  & \offset Security of Distributed Systems (incl. blockchains): \cite{BVP_CLOUD11,BVB_SiiS11,DRTV_FoundationCoding15,DRTV_ThCode18,IVP_ICOIN18,DLTV_Blockchains18,DLTV_Blockchains19,DLTV_Blockchains22,DLRTV_NFT22}\\
                                  & \offset Crash/cheating faults tolerance \cvcite{VRL_SBAC04,KRJV_EGC05, RV_Pasco07,Var_phD07,GGPV_PDP09,MVBSK_CAMWA12,MVB_CEC2013,MVJB_NSS14,MVB_Evostar2016}\\
                                  & \offset Code Obfuscation \cvcite{VTB_NIDISC13,BVB_NSS13}, Secure IaaS Cloud \cvcite{BVP_CLOUD11, BVB_Renpar11, BVB_TSI12} \\

                                  %& \offset  Design of the authentication system of Grid'5000
                                  % \cvcite{VGMRL_Gada05}\\
                                  %& \offset  Developing security awareness by education \cvcite{DRTV_ThCode07,BCCDV_DistSyst11_Chap11,DRTV_FoundationCoding15,DLTV_Blockchains18}
                                  % & \offset \offset \underline{\textit{Keywords:}} Grid Computing, Cloud, Security,
                                  % Fault-Tolerance, Authentication\\
                                  % \\
                                  % \multicolumn{2}{l}{I am a quick learner and a good communicator with the
                                  % ability to work within multi-cultural teams.}
  }
  \textsc{Books} \ifShortOrFull{\cite{VB_Prog_C07,DRTV_FoundationCoding15,DRTV_ThCode18,DLTV_Blockchains22,DLRTV_NFT22}}:
                                  &\\
                                  & \vspace*{-1em}
                                    \href{http://varrette.gforge.uni.lu/books/programmation-avancee-c/}{\includegraphics[height=8em]{cover-programmation-avancee-c.jpg}}
                                    \href{http://varrette.gforge.uni.lu/books/theorie-des-codes/}{\includegraphics      [height=8em]{cover-theorie-des-codes.jpg}}
                                    \href{http://varrette.gforge.uni.lu/books/foundations-of-coding/}{\includegraphics  [height=8em]{cover-foundations-of-coding.jpg}}
                                    \href{http://varrette.gforge.uni.lu/books/blockchains/}{\includegraphics            [height=8em]{cover-blockchains.png}}
                                    \href{http://varrette.gforge.uni.lu/books/nft/}{\includegraphics                    [height=8em]{cover-nft.png}}
  \\
\end{tabular}
\ifShortOrFull{\vspace{-1.5em}}
%}

% ---------- Education -----------
\ifShortOrFull{
  % ===============================================================================
% _education.tex -- my personal education
% 
% Copyright (c) 2009-2011 Sebastien Varrette <Sebastien.Varrette@uni.lu>
% .             http://varrette.gforge.uni.lu
% 
% This file is part of my CV (see 'cv-varrette-en.tex')
%
% This work is licensed under the Creative Commons
% Attribution-NonCommercial-ShareAlike 3.0 Unported License (see LICENCE file)
% ===============================================================================

\begin{rubriquetableau}[\offsetintab]{Education}
    2007
    & \textsc{Ph.D. in Computer Science},
    \hfill with honours  (\textsl{Excellent})\\
    & \vspace{-0.8em}\acf{UL} \& \acf{INPG}\\
    & \offset Thesis: \emph{Security in Large Scale Distributed Systems:
      Authentication and Result Checking}\\
    & \offset Advisors\,: Franck Lepr�vost (\ac{UL}) \& Jean-Louis Roch (\ac{INPG})\\
    \\
    2003
    & \textsc{M.Sc. in Computer Science} \hfill with honours (\textsl{TB/First Class})\\
    & Speciality: Cryptology, Security and Information Coding (CSCI), \hfill rank: 1st\\
    & \acf{INPG} \& \acf{UJF}\\
    & \offset Thesis: \emph{Securing Peer-to-Peer Computing in Grid Environments}\\
    & \offset Advisor: Jean-Louis Roch (\ac{INPG}) \\
    \\
    2003
    & \textsc{Master's Degree in Engineering}
    (\TelecomsENSIMAG)\hfill with honours (\textsl{B/2.1})\\
    & Speciality: Computer Sciences and Telecommunications \hfill rank: top 10\%\\
%   \\
    1997 -- 2000
    & \textsc{Advanced Mathematics and Physics studies} \\
    & \offset For the competitive entrance to the French engineering schools
    \\
    1997
    & \textsc{Scientific "Baccalaur�at"} (French "A" levels) \hfill with honours (\textsl{B/2.1})
\end{rubriquetableau}


% ===============================================================================
% eof
% 
% Local Variables:
% fill-column: 80
% mode: latex
% mode: flyspell
% mode: auto-fill
% End:
 % ALL
}

% ---------- Employment -----------
\iffullcv{%\ifShortOrFull{
  \vspace*{-1.5em}
  % ==============================================================================
% _employment.tex -- Details of my employments
% 
% Copyright (c) 2009-2011 Sebastien Varrette <Sebastien.Varrette@uni.lu>
% .             http://varrette.gforge.uni.lu
% 
% This file is part of my CV (see 'cv-varrette-en.tex')
%
% This work is licensed under the Creative Commons
% Attribution-NonCommercial-ShareAlike 3.0 Unported License (see LICENCE file)
% ==============================================================================

\begin{rubriquetableau}[\offsetintab]{Employment}
    2007 -- now  & \textsc{Scientific collaborator} (\UL, Luxembourg)\\
    2004 -- 2007 & \textsc{Research Assistant} (\UL, Luxembourg)\\
    2003         & \textsc{System administrator} (\ID, Grenoble, France)
\end{rubriquetableau}

% ==============================================================================
% eof
%
% Local Variables:
% mode: latex
% mode: flyspell
% mode: auto-fill
% fill-column: 80
% End:

}

% ---------- Awards -----------
\ifShortOrFull{
  \vspace*{-0.5em}
  % Time-stamp: <Tue 2020-01-07 10:32 svarrette>
% ==============================================================================
% _awards.tex --
%
% Copyright (c) 2018 Sebastien Varrette <Sebastien.Varrette@uni.lu>
% .             http://varrette.gforge.uni.lu
%
% ==============================================================================
% This file is part of my CV (see 'cv-varrette-en.tex' and README)
%
% This work is licensed under the terms of the Creative Commons CC-by-nc-sa 3.0
% licence (see LICENCE). For more details, visit:
% .         http://creativecommons.org/licenses/by-nc-sa/3.0/
% ==============================================================================

\begin{rubriquetableau}[\offsetintab]{Awards}
  2018  & Best Paper Award, 32$^\text{th}$ IEEE Intl. Conf. on Information Networking (ICoin 2018)\hfill\cvcite{IVP_ICOIN18}\\
  2014  & Best Student Paper Award, 8$^\text{th}$ IEEE Intl. Conf. on Network \& System Security (NSS 2014)\hfill\cvcite{MVJB_NSS14}\\
\end{rubriquetableau}

% ==============================================================================
% eof
%
% Local Variables:
% mode: latex
% mode: flyspell
% fill-column: 80
% End:

}
\unlesstinycv{
  \iffullcv{\clearpage}
  % ---------- Teaching-----------
  \ifShortOrFull{
    % Time-stamp: <Sat 2018-04-07 00:10 svarrette>
% ==============================================================================
% _teaching.tex -- details of the lectures carried out
%
% Copyright (c) 2009-2011 Sebastien Varrette <Sebastien.Varrette@uni.lu>
% .             http://varrette.gforge.uni.lu
%
% ==============================================================================
% This file is part of my CV (see 'cv-varrette-en.tex' and README)
%
% This work is licensed under the terms of the Creative Commons CC-by-nc-sa 3.0
% licence (see LICENCE). For more details, visit:
% .         http://creativecommons.org/licenses/by-nc-sa/3.0/
% ==============================================================================

\begin{rubriquetableau}[\offsetintab]{Teaching Experience}
  2014 -- now  & \textsc{UL HPC Schools} \hfill \url{https://hpc.uni.lu/hpc-school/} \\
  2008 -- now  & \textsc{Parallel \& Grid Computing} %, \textsc{Middleware}
  \hfill Master MICS2 (UL)\\
  % \\
  2004 -- 2007 & \textsc{Programming Techniques I}   \hfill Bachelor I1/CUT1 (UL)\\
  & \textsc{Advanced Programming in C, C++ and Java} \hfill Bachelor I2 (UL)\\
  & \textsc{System Administration and Network Security} \hfill Master CSCI2 (UJF)\\
  % \\
  2006         & \textsc{Cryptology and Network Security}\hfill
  Master (\href{http://www.uy1.uninet.cm/}{Univ. of Yaound� I}, Cameroon)
\end{rubriquetableau}

% ==============================================================================
% eof
%
% Local Variables:
% mode: latex
% mode: flyspell
% mode: auto-fill
% fill-column: 80
% End:

  }

  % ---------- Additional information ----------
  \iffullcv{
    % Time-stamp: <Wed 2012-03-07 09:31 svarrette>
% ===============================================================================
% _other.tex -- other information
%
% Copyright (c) 2009-2011 Sebastien Varrette <Sebastien.Varrette@uni.lu>
% .             http://varrette.gforge.uni.lu
%
% ==============================================================================
% This file is part of my CV (see 'cv-varrette-en.tex' and README)
%
% This work is licensed under the terms of the Creative Commons CC-by-nc-sa 3.0
% licence (see LICENCE). For more details, visit:
% .         http://creativecommons.org/licenses/by-nc-sa/3.0/
% ==============================================================================

\begin{rubriquetableau}[\offsetintab]{Additional Information}
    % & Driver's Licence\\
    \textbf{Languages}
    & \textsc{French} (native language), \textsc{English}
    (fluent) and \textsc{German} (basic knowledge)
    \\
    \textbf{Sports}
    & Karate (Black belt -- {\small 3rd DAN}, \acf{DIF}), Jogging, Badminton, Ski
    etc.
    \\
    % \textbf{Hobbies}
    % & Collectible Card Games (VTES), Video Games etc.\\
\end{rubriquetableau}

% ===============================================================================
% eof
%
% Local Variables:
% mode: latex
% mode: flyspell
% mode: auto-fill
% fill-column: 80
% End:

  }
  %\ifshortcv{\clearpage}
  %\ifshortcv{% Time-stamp: <Mon 2018-08-20 21:41 svarrette>
% ===============================================================================
% _phd_board.tex -- Participation to PhD boards
%
% Copyright (c) 2009-2018 Sebastien Varrette <Sebastien.Varrette@uni.lu>
% .             http://varrette.gforge.uni.lu
%
% ==============================================================================
% This file is part of my CV (see 'cv-varrette-en.tex' and README)
%
% This work is licensed under the terms of the Creative Commons CC-by-nc-sa 3.0
% licence (see LICENCE). For more details, visit:
% .         http://creativecommons.org/licenses/by-nc-sa/3.0/
% ==============================================================================

\begin{rubriquetableau}[\offsetintab]{Participation to Ph.D Boards}
  Oct. 2016 & \textsc{Alban Rousset}, Institut FEMTO-ST, Université de Franche-Comté\\
  & \emph{Contribution à la distribution et à la synchronisation des Systèmes Multi-Agents sur les super-calculateurs}, Examinateur
  \\
  Sept. 2018 & \textsc{Gabriele Pozzetti}, University of Luxembourg\\
  & \emph{A dual-grid  multiscale  approach  tocfd-demcouplings  for  multiphase  flow}, Jury member
  \\
  Dec. 2018 & \textsc{David Guyon}, University of Rennes I\\
  & \emph{Energy-efficient Cloud Elasticity for Data-driven Applications} (pending exact title), Jury Member
  \\
\end{rubriquetableau}


% ===============================================================================
% eof
%
% Local Variables:
% mode: latex
% mode: flyspell
% mode: visual-line
% fill-column: 80
% End:
}

  % ---------- Graduate student supervision -----------
  % Time-stamp: <Tue 2020-01-07 11:11 svarrette>
% ===============================================================================
% _supervision.tex -- Students I used to supervise
%
% Copyright (c) 2009-2011 Sebastien Varrette <Sebastien.Varrette@uni.lu>
% .             http://varrette.gforge.uni.lu
%
% ==============================================================================
% This file is part of my CV (see 'cv-varrette-en.tex' and README)
%
% This work is licensed under the terms of the Creative Commons CC-by-nc-sa 3.0
% licence (see LICENCE). For more details, visit:
% .         http://creativecommons.org/licenses/by-nc-sa/3.0/
% ==============================================================================

\begin{rubriquetableau}[\offsetintab]{Graduate Students Supervision}
  \textbf{PostDocs.}
  & \supervision{Ezhilmathi Krishnasamy}{2019 --}{PRACE-6IP project coordination, advanced HPC/research support}\\
  & \supervision{Emmanuel Kieffer}{2019 --}{Bi-level optimization and scalable science}\\
  & \supervision{Alban Rousset}{2016 -- 2018}{Large scale parallel simulation for
    Discrete Element Method}
  \\
  & \supervision{Joseph Emeras}{2014 -- 2016}{Workload Analysis and
    Characterization of HPC Platforms for the Study of Virtualization Service}
  \\
  \textbf{PhD.}
  & \supervision{Ludovica Paseri}{2019 -- }{GDPR compliance in Eurorean HPC and Cloud Computing Initiatives}
    \\
  & \supervision{Abdallah A.Z.A. Ibrahim}{2017 -- 2020}{\textsc{Presence}: Toward a Novel Approach
    for Performance Evaluation of Mobile Cloud SaaS Web Services}
  \\
  & \supervision{Abdoul-Wahid C. Mainassara}{2017 -- }{Large scale parallel
    simulation for Discrete Element Method}
  \\
  & \supervision{Chao Liu}{2017 -- }{Pricing strategies for cloud brokers at the Software-as-a-Service (SaaS) level}
  \\
  & \supervision{Jakub Muszy\'nski}{2011 -- 2015}{Cheating-Tolerance of Parallel and Distributed Evolutionary Algorithms in Desktop Grids and Volunteer Computing Systems}
  \\
  & \supervision{Beno\^it Bertholon}{2010 -- 2013}{CertiCloud \& JShadObf:
    Toward Integrity and Software Protection in Cloud Computing Platforms}
  \\
  \ifSmallOrShort{
    \emph{In addition:} & 22 master and 5 bachelor students supervision for
    their last year project and training.\\
    & Team leader (deputy head), Core management team of the Uni.lu HPC facility (Total: 8 FTEs).
  }
  \iffullcv{
    \\
    \textbf{Master}
    &
    \supervision{Abatcha Olloh}{2020 -- }{Infrastructure and HPC Architect Engineer of the 
      \href{http://hpc.uni.lu}{UL HPC facility}}
    \\
    &
    \supervision{Teddy Valette}{2020 -- }{Infrastructure and HPC Architect Engineer of the 
      \href{http://hpc.uni.lu}{UL HPC facility}}
    \\
    &
      \supervision{Kevin Fornasiero}{2020}{Performance evaluation, energy efficiency and automatic testing of accelerated HPC applications} \\
    &
    \supervision{Tazio Gennuso}{2019}{Analyse de la consommation des ressources des syst\`emes HPC (ENSIIE)} \\
    &
    \supervision{Sean Mahon}{2019}{Performance Analysis of Distributed and Scalable Deep Learning (SoHPC)} \\
    &
    \supervision{Cl\'ement Courageux-Sudan}{2019}{Blockchains environments to support IoT
      developments} \\
    &
    \supervision{Cl\'ement Parisot}{2017 -- 2019}{System administrator of the
      \href{http://hpc.uni.lu}{UL HPC facility}}
    \\
    & \supervision{Maxime Schmitt}{2014--2015}{RESIF and OpenStack Deployment}
    \\
    & \supervision{Sarah Peter (born Diehl)}{2014 --}{System administrator of the
      \href{http://hpc.uni.lu}{UL HPC facility}}
    \\
    & \supervision{Ludovic Schoepps}{2014}{Design a REST API service to Monitor UL HPC Resources}
    \\
    & \supervision{Anna Giannakou}{2013}{Energy Efficiency in HPC environments}
    \\
    & \supervision{Valentin Plugaru}{2012 --}{Energy Efficiency of Hypervisors
      and Cloud middleware}
    \\
    & \supervision{S\'ebastien Martinez}{2012}{Source Code Obfuscation by mean
      of Evolutionary Algorithms}
    \\
    & \supervision{Fotis Georgatos}{2012 -- 2014} {System administrator of the
      \href{http://hpc.uni.lu}{UL HPC facility}}
    \\
    & \supervision{Mateusz Guzek}{2011}{System administration of cluster-based HPC systems}
    \\
    & \supervision{Beno\^it Bertholon}{2009}{Integrity issues in distributed executions}
    \\
    & \supervision{Christophe Weis}{2009}{Optimizing hash functions and S-Box by GEP}
    \\
    & \supervision{Dominic Dunlop}{2008 -- 2009}{Using GAs to tune benchmarks for HPCs}
    \\
    & \supervision{Romain Cavagna}{2007 -- 2008}{Deployment of a GForge platform}
    \\
    \textbf{Bachelor}
    & \supervision{Desislava Marinova}{2018}{Cloud Computing Security}
    \\
    & \supervision{Anthony Mathieu}{2016}{Syslogd and centralized log
      management}
    \\
    & \supervision{Derick A. L. Ramirez}{2016}{Energy monitoring on the UL HPC
      facility}
    \\
    & \supervision{Hyacinthe Cartiaux}{2011 -- } {System administrator of the
      \href{http://hpc.uni.lu}{UL HPC facility}}
    \\
    & \supervision{Bernard Reichert}{2008}{Collision detection implementation
      using CUDA}
  }

\end{rubriquetableau}

% ===============================================================================
% eof
%
% Local Variables:
% mode: latex
% mode: flyspell
% mode: auto-fill
% fill-column: 80
% End:

  \vspace{-0.5em}
  \iffullcv{\clearpage}
  % ---------- Participation PhD. Board -----------
  \iffullcv{% Time-stamp: <Mon 2018-08-20 21:41 svarrette>
% ===============================================================================
% _phd_board.tex -- Participation to PhD boards
%
% Copyright (c) 2009-2018 Sebastien Varrette <Sebastien.Varrette@uni.lu>
% .             http://varrette.gforge.uni.lu
%
% ==============================================================================
% This file is part of my CV (see 'cv-varrette-en.tex' and README)
%
% This work is licensed under the terms of the Creative Commons CC-by-nc-sa 3.0
% licence (see LICENCE). For more details, visit:
% .         http://creativecommons.org/licenses/by-nc-sa/3.0/
% ==============================================================================

\begin{rubriquetableau}[\offsetintab]{Participation to Ph.D Boards}
  Oct. 2016 & \textsc{Alban Rousset}, Institut FEMTO-ST, Université de Franche-Comté\\
  & \emph{Contribution à la distribution et à la synchronisation des Systèmes Multi-Agents sur les super-calculateurs}, Examinateur
  \\
  Sept. 2018 & \textsc{Gabriele Pozzetti}, University of Luxembourg\\
  & \emph{A dual-grid  multiscale  approach  tocfd-demcouplings  for  multiphase  flow}, Jury member
  \\
  Dec. 2018 & \textsc{David Guyon}, University of Rennes I\\
  & \emph{Energy-efficient Cloud Elasticity for Data-driven Applications} (pending exact title), Jury Member
  \\
\end{rubriquetableau}


% ===============================================================================
% eof
%
% Local Variables:
% mode: latex
% mode: flyspell
% mode: visual-line
% fill-column: 80
% End:
}

  %\ifsmallcv{\clearpage}

  % ---------- Research Project ------------
  %\iffullcv{\clearpage}
  % Time-stamp: <Sat 2018-04-07 23:59 svarrette>
% ===============================================================================
% _research_project.tex -- Research projects I'm involved in (at various degrees)
%
% Copyright (c) 2009-2011 Sebastien Varrette <Sebastien.Varrette@uni.lu>
% .             http://varrette.gforge.uni.lu
%
% ==============================================================================
% This file is part of my CV (see 'cv-varrette-en.tex' and README)
%
% This work is licensed under the terms of the Creative Commons CC-by-nc-sa 3.0
% licence (see LICENCE). For more details, visit:
% .         http://creativecommons.org/licenses/by-nc-sa/3.0/
% ==============================================================================

\begin{rubriquetableau}[\offsetintab]{Research Projects}
    2007 -- now  & \href{http://hpc.uni.lu}{UL High Performance Computing (HPC)}, co-PI \hfill (UL cumulative contribution: \textbf{\ulhpcCumulInvestment\ \euro{}}) \\
    2014 -- 2018 & EU \href{http://www.cost.eu/domains_actions/ict/Actions/IC1305/}{\textsc{COST Action IC1305}}:
    {\small Network for Sustainable Ultrascale Computing (NESUS)}
    \\
    2016 -- 2019 & co-PI, UL \textsc{Lsdem} (UL contribution: 332 k\euro{})
    \\
    2014 -- 2016 & AFR PostDoc \textsc{J. Emeras} (\textsc{Evalix}; Co-Supervisor; Total/AFR contribution:
    56 k\euro{})\\
    & \offset
    {\small Workload Analysis \& Characterization of HPC Platforms for the Study of Virtualization Service}\\
    2011 -- 2013 & UL \textsc{EvoPerf} (UL contribution: 373 k\euro{})
    \\
    2010 -- 2012 &
    FNR CORE \href{http://greenit.gforge.uni.lu/}{\textsc{GreenIT}}
    (Total: 1,5 M\euro{}, FNR contribution: 432 k\euro{})
    % {\small holistic autonomic energy-efficient solution to manage, provision, and
    %   administer CC/HPC data centers} % WP7
    \\
    2010 -- 2012 & AFR PhD \textsc{B. Bertholon}
    (PHD-09-142; Scientific Advisor; Total/AFR contribution:
    110 k\euro{})\\
    & \offset
    {\small Confidentiality and Integrity Issues over Cloud Computing Platforms}
    \\
    2009 -- 2013 & EU \href{http://www.cost804.org/}{\textsc{COST Action IC0804}}:
    {\small Energy efficiency in large scale distributed systems}
    \\
    2009 -- 2013 &
    EU \href{http://www.coregrid.net/}{\textsc{CoreGrid}}
    % {\small Foundations, Software Infrastructures and Applications for large
    %   scale distributed, Grid and P2P Technologies (Trust and Security project)}
    \\
   2006 -- 2008 &
    ANR \href{http://www-lipn.univ-paris13.fr/safescale/}{\textsc{SafeScale-BGPR}}
    (ANR-05-SSIA-0005; ANR contribution: 68 k\euro{})
    % {\small Security \& Fault-tolerance to Exploit Safety ambient Computing in lArge
    %   scaLe Environments}
    \\
    2005 -- now & \href{http://www.grid5000.org}{\textsc{Grid'5000}} (technical committee)
    % {\small highly reconfigurable, controlable and monitorable experimental
    %   grid}
    \\
    2005 -- 2007 & FNR-SECOM \textsc{TeseGrad} (FNR contribution: 300 k\euro{})
    %{\small Techniques for Securing Grids and Ad-Hoc networks project}
    \\
    2004 -- 2007 &
    \href{http://www-fourier.ujf-grenoble.fr/~gillard/cryptalpes.html}{\textsc{CryptAlpes}}
    % {\small mathematical aspects of cryptography; security aspects of
    %   distributed/P2P computations}
    \\
    2004 -- 2006 & %Regional % PTP % Projet Thematique Prioritaire
    \href{http://infographie.univ-lyon2.fr/~miguet/ragtime}{\textsc{RagTime}}
    (Total: 545 k\euro{}, Rh\^one-Alpes Region contribution: 217 k\euro{})
    % {\small Rh\^one-Alpes : Grille pour le Traitement d'Informations M�dicales
    %   (WP3: Acc\`es aux Donn\'ees et S\'ecurit\'e)}
    \\
\end{rubriquetableau}

% ===============================================================================
% eof
%
% Local Variables:
% mode: latex
% mode: flyspell
% mode: visual-fill
% End:


  % ---------- Professional development ----------
  \ifShortOrFull{
    % Time-stamp: <Mon 2011-06-06 15:53 svarrette>
% ===============================================================================
% _professional_dev.tex -- professionnal development
% 
% Copyright (c) 2009-2011 Sebastien Varrette <Sebastien.Varrette@uni.lu>
% .             http://varrette.gforge.uni.lu
% 
% ==============================================================================
% This file is part of my CV (see 'cv-varrette-en.tex' and README)
% 
% This work is licensed under the terms of the Creative Commons CC-by-nc-sa 3.0
% licence (see LICENCE). For more details, visit:
% .         http://creativecommons.org/licenses/by-nc-sa/3.0/
% ==============================================================================

\begin{rubriquetableau}[\offsetintab]{Professional Development}
    2010 -- now  & \textsc{Technical advisor} for \href{http://lcsb.uni.lu}{LCSB}: %\\
    %& \offset
    {\small Design \& implementation of BioCore HPC facilities}
    \\
    2009 -- now  & \textsc{Technical advisor} for Luxembourg's \href{http://www.eco.public.lu/}{Ministry of Economy and External Commerce}\\
    & \offset \emph{Project}: Design of an HPC facility in Luxembourg \\
    2007 -- now  & \textsc{System Administrator} for the UL's HPC facilities which mainly
    include currently: \\
    &
    \offset \offset
    \href{http://gridlux.gforge.uni.lu/resources.html#chaos_cluster}{Chaos cluster}:
    67 nodes, 656 computing cores, $R_{\text{peak}} = 6,125$ TFlops
    \\
    & \offset \offset
    \href{http://gridlux.gforge.uni.lu/resources.html#lux5000_cluster}{Lux5000 cluster}:
    22 nodes, 176 computing cores, $R_{\text{peak}} = 1,4$ TFlops, part of \href{http://www.grid5000.fr}{Grid5000}
    \\
    2007 -- now  & \textsc{System Administrator} for various CSC services (Gforge,
    wikis, etc.)
    \\
    2006         & \href{http://www.egide.asso.fr/}{EGIDE} mission for 3 weeks
    in Cameroon (master lecture -- Univ. of Yaoud� I)\\
    2005 & \textsc{Editorial Committee}, UL new degree \\
    & \offset $\bullet$ Master of Science \emph{Security and Trust} and \emph{Security management} \\
    & \offset $\bullet$ Bachelor \emph{Informatique de Gestion}
    \\
    2005 -- 2006 & \textsc{Editorial Committee}, UL website (CMS selection \& management)\\
    2005 -- 2006 & \textsc{Faculty Council}, elected member (assistant representative)\\
    2000 -- 2002 & \textsc{Ensimag Junior Enterprise}, developer member\\
\end{rubriquetableau}


% ===============================================================================
% eof
% 
% Local Variables:
% fill-column: 80
% mode: latex
% mode: flyspell
% End:

    \ifshortcv{\clearpage}
  }

  % ---------- Participation -----------
  % % Time-stamp: <Wed 2012-03-07 08:22 svarrette>
% ===============================================================================
% _participation.tex -- Participation to/animation of the research community
% 
% Copyright (c) 2009-2011 Sebastien Varrette <Sebastien.Varrette@uni.lu>
% .             http://varrette.gforge.uni.lu
% 
% ==============================================================================
% This file is part of my CV (see 'cv-varrette-en.tex' and README)
% 
% This work is licensed under the terms of the Creative Commons CC-by-nc-sa 3.0
% licence (see LICENCE). For more details, visit:
% .         http://creativecommons.org/licenses/by-nc-sa/3.0/
% ==============================================================================

% \begin{rubriquetableau}[\offsetintab]{Participation to PhD supervision}
%     2011 -- 2013 & Thesis of Jakub Muszy\'nski, \UL (Scientific Advisor)
%     %     \\
%     %     & \offset
%     Subject: \emph{Integrity aspects on grid and cloud platforms}
%     \\
%     2010 -- 2012 & Thesis of Beno\^it Bertholon, \UL (Scientific Advisor)
%     %     \\
%     %     & \offset
%     Subject: \emph{Integrity \& confidentiality issues on the Cloud}
%     \\
% \end{rubriquetableau}

% \begin{rubriquetableau}[\offsetintab]{Participation to PhD boards}

% \end{rubriquetableau}

\begin{rubriquetableau}[\offsetintab]{Participation to the research community}
    % \textbf{Affiliations}
    % & \acf{CSC} Research Unit (\UL, Luxembourg) \hfill 2007 -- now\\
    % & \acf{LACS} (\UL, Luxembourg) \hfill 2004 -- 2007\\
    % & \acf{ID} (Grenoble, France) \hfill 2003 -- 2007\\
    \textbf{Membership}
    & IEEE Computer and Computational Intelligence society
    \\
    \textbf{Events}
    %%%%%%%%%%%%%%%%%%%%%%%%%%%%%%%%%%%%%%%%%%%%%%%%% 
    & \textsc{General (co-)chair}: %\\ & \offset
    \conference{LPWST 2010}{http://lpwst2010.uni.lu}
    {1st Luxembourg-Polish Workshop on Security \& Trust}{Luxembourg}{May 2010}
    \\
    %%%%%%%%%%%%%%%%%%%%%%%%%%%%%%%%%%%%%%%%%%%%%%%%% 
    & \textsc{Program or Organizational Chair}: \\ & \offset
    % 2011
    \conference{OPTIM 2011}
    {http://hpcs11.cisedu.info/conference/workshops/workshop-03---optim}
    {Workshop on Optimization Issues in Energy Efficient Distributed Systems}
    {Istanbul, Turkey}{July 2011},
    % 
    \conference{SCALSOL 2011}
    {http://scalsol.gforge.uni.lu/}
    {Workshop on Scalable Solutions for GreenIT}
    {Pafos, Cyprus}
    {Sept 2011},
    % 2010
   \conference{OPTIM 2010}
    {http://cisedu.us/cis/hpcs/10/main/storageDocs.jsp?doc=/docs/hpcs/10/workshops/W3.Optim.html}
    {Workshop on Optimization Issues in Energy Efficient Distributed Systems}
    {Caen, France}{June 2010}
    \\
    % & \textsc{Session Organizer}: \\ & \offset
    % \\
    % & \textsc{Session Chair}: \\ & \offset
    % \\
    %%%%%%%%%%%%%%%%%%%%%%%%%%%%%%%%%%%%%%%%%%%%%%%%% 
    & \textsc{International Program Committee Member}: \\ & \offset
    % 2011
    \conference{PCGrid 2011}
    {http://pcgrid.imag.fr/}
    {5th Workshop on Desktop Grids and Volunteer Computing Systems}
    {Anchorage, USA}
    {May 2011},
    % 2010
    \conference{PCGrid 2010}
    {http://mescal.imag.fr/membres/pcgrid2010/}
    {4th Workshop on Desktop Grids and Volunteer Computing Systems}
    {Melbourne, Australia}
    {May 2010}
    \\
    %%%%%%%%%%%%%%%%%%%%%%%%%%%%%%%%%%%%%%%%%%%%%%%%% 
    \textbf{Reviews} %%%%%%%%%%%%%%%%%%%%%%%%%%%%%%%%
    %%%%%%%%%%%%%%%%%%%%%%%%%%%%%%%%%%%%%%%%%%%%%%%%%     
    & \textsc{Journal reviewer}: %\\ & \offset
    % Future Generation Computer Systems
    \journal{FGCS}{http://ees.elsevier.com/fgcs/}{Elsevier},
    % J. of Supercomputing
    \journal{J. of Supercomputing}{http://www.editorialmanager.com/supe/}{Springer},
    % J. of Parallel and Distributed Computing
    \journal{JPDC}{http://www.elsevier.com/locate/jpdc}{Elsevier},
    %\\ & \offset
    % J. of Parallel Computing
    \journal{PARCO}{http://www.elsevier.com/locate/parco}{Elsevier},
    % J. of Foundations of Computer Science
    \journal{IJFCS}{http://www.worldscinet.com/ijfcs/ijfcs.shtml}{WO},
    % IEEE Transactions on Parallel and Distributed Systems
    \journal{TPDS}{http://www.computer.org/portal/web/tpds/}{IEEE}
    \\
    %%%%%%%%%%%%%%%%%%%%%%%%%%%%%%%%%%%%%%%%%%%%%%%%% 
    & \textsc{International Conference reviews} (in addition to the above
    elements \& events): \\ & \offset
    % 2011
    \conference{EuroPar 2011}
    {http://europar2011.bordeaux.inria.fr/}
    {}
    {Bordeaux, France}
    {2011},
    \conference{SiiS 2011}
    {http://siis.ipipan.waw.pl/}
    {International Joint Conference Security and Intelligent Information
      Systems}
    {Warsaw, Poland}
    {2011},
    % 2010
    \conference{MENS 2010}
    {http://www.efipsans.org/mens2010/}
    {2nd IEEE International Workshop on Management of Emerging Networks and
      Services}
    {Miami, USA}
    {2010}
    \conference{NAS 2010}
    {http://www.nas-conference.org/NAS-2010/}
    {5th IEEE International Conference on Networking, Architecture, and Storage}
    {Macau, China}
    {2010}
    % 2008
    \conference{HPCS 2008}
    {http://cisedu.us/cis/hpcs/08/index.html}
    {High Performance Computing & Simulation Conference}
    {Cyprus}{2008},
    % 
    \conference{MCO'08}
    {http://www.lita.univ-metz.fr/~mco08/}
    {Modelling, Computation and Optimization in Information Systems and Management Sciences}
    {Metz, France}{2008}
    % 2007
    \conference{PSAI 2008}
    {http://cpsr.org/news/07news/psai08/}
    {Workshop on Privacy and Security - Artificial Intelligence}
    {Barcelona, Spain}
    {2008}
    % 
    \conference{GPC 2007}
    {http://www-lipn.univ-paris13.fr/GPC2007/}
    {Intl. Conf. on Grid and Pervasive Computing}
    {Paris, France}
    {2007},
    % 2004
    \conference{HICSS-38}
    {http://www.hicss.hawaii.edu/hicss38/apahome38.html}
    {Hawaii International Conference on System Sciences}
    {Hawaii, USA}
    {2005}
\end{rubriquetableau}


% ===============================================================================
% eof
% 
% Local Variables:
% fill-column: 80
% mode: latex
% mode: flyspell
% End:

}
\iffullcv{\clearpage}
% ---------- Publications ----------
% Time-stamp: <Tue 2021-07-27 16:10 svarrette>
% ==============================================================================
% _publis.tex -- List of my personal publication
%
% Copyright (c) 2009-2011 Sebastien Varrette <Sebastien.Varrette@uni.lu>
% .             http://varrette.gforge.uni.lu
%
% ==============================================================================
% This file is part of my CV (see 'cv-varrette-en.tex' and README)
%
% This work is licensed under the terms of the Creative Commons CC-by-nc-sa 3.0
% licence (see LICENCE). For more details, visit:
% .         http://creativecommons.org/licenses/by-nc-sa/3.0/
% ==============================================================================
%
% Based on the files produced by 'scripts/split_bibtex_per_type.pl'
%

\def\bibfile{biblio-varrette}

\begin{rubrique}{Publications}
  % Summary of the publications
  \IfFileExists{__sub_\bibfile_summary.tex}{
    \input{__sub_\bibfile_summary}
  }{}

  % Publish or Perish statistics (based on Google Scholar)
  % You can run it on Mac using PlayOnMac
  {\footnotesize
    % TODO on UPDATE:
    % search "varrette" in the domain "Engineering, Computer Science, Mathematics"

    % varrette
    \vspace*{-2em}
    \begin{center}
      \begin{tabular}{|c||cccc|}
        \hline
        \multirow{2}{4em}{\href{http://www.harzing.com/pop.htm}{Publish or Perish}}
        & Papers:186        & Citations:984,Years:18   & \textbf{h-index:13} & g-index:28 \\
        & Cites/year: 54.67 & Cites/paper: 5.29        & Authors/paper: 3.79 & Query date: \textbf{2021-07-27}\\
        \hline
        \href{https://orbilu.uni.lu/simple-search?query=varrette}{Orbi$^\text{lu}$} &
        \multicolumn{1}{c|}{\href{http://www.informatik.uni-trier.de/~ley/pers/hd/v/Varrette:S=eacute=bastien.html}{DBLP}} &                                                                                                \multicolumn{1}{c|}{\href{https://scholar.google.fr/citations?hl=fr\&user=6PTStIcAAAAJ\&view_op=list_works\&sortby=pubdate}{Google Scholar}}\\
        \cline{1-3}
      \end{tabular}
    \end{center}
  }
  %~\\[2em]

  % Detailed list upon FULL cv
  \iffullcv{
    \IfFileExists{__sub_\bibfile_main.tex}{
      \input{__sub_\bibfile_main}
    }{}
  }

  % \iftoggle{fullcv}{Full CV}{}
  % \iftoggle{smallcv}{Small CV}{}
  % \iftoggle{shortcv}{Short CV}{}
  % \iftoggle{tinycv}{Tiny CV}{}

  % Selected list for short and small CV
  \ifSmallOrShort{
    \vspace{-1em}
    \paragraph{Selected Publications}~\\
    \IfFileExists{__sub_selected_\bibfile_main.tex}{\input{__sub_selected_\bibfile_main}}{}
  }

\end{rubrique}





% ==============================================================================
% eof
%
% Local Variables:
% mode: latex
% mode: flyspell
% fill-column: 80
% End:



% ===========  Acronyms ===========
\begin{acronym}
  \acro{BD}{Big Data}
  \acro{CC}{Cloud Computing}
  \acro{CSC}[\href{http://csc.uni.lu}{CSC}]{Computer Science and Communications}
  \acro{DIF}[\href{http://www.ffkama.fr/direction-technique/formation/formationsfederales.php}{DIF}]{Federal Instructor Diploma}
  \acro{EA}{Evolutionary Algorithm}
  \acro{ETP4HPC}{European Technology Platform (ETP) in the area of HPC}
  % \acro{ENSIMAG}[\href{http://ensimag.grenoble-inp.fr/}{ENSIMAG}]{}
  \acro{HPC} {High Performance Computing}
  \acro{INPG}[\href{http://www.grenoble-inp.fr/}{INPG}]{Institut National Polytechnique de Grenoble}
  \acro{IaaS}{Infrastructure-as-a-Service}
  \acro{ID}[\href{http://www-id.imag.fr}{ID-IMAG}]{Laboratoire Informatique et Distribution}
  \acro{LACS}[\href{http://lacs.uni.lu}{LACS}]{Laboratory of Algorithmics, Cryptology and Security}
  \acro{NESUS}{Network for Sustainable Ultrascale Computing}
  \acro{NVAITC}{NVidia Artificial Intelligence (AI) Tech. Center}
  \acro{PRACE}[\href{http://www.prace-ri.eu/}{PRACE}]{Partnership for Advanced Computing in Europe}
  \acro{TC}{Trusted Computing}
  \acro{TCG}{Trusted Computing Group}
  \acro{TPM}{Trusted Platform Module}
  \acro{UJF}[\href{http://www.ujf-grenoble.fr/}{UJF}]{University Joseph Fourier}
  \acro{UL}[\href{http://www.uni.lu}{UL}]{University of Luxembourg}
\end{acronym}

\end{document}
% ==============================================================================
% eof
%
% Local Variables:
% mode: latex
% mode: flyspell
% mode: auto-fill
% fill-column: 80
% End:
